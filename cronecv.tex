\documentclass[11pt]{res}

%packages
\usepackage{setspace}
\usepackage[letterpaper]{typearea}

% footer w/ page numbers
\usepackage{lastpage}
\usepackage{fancyhdr}
\fancyhf{} % sets both header and footer to nothing
\renewcommand{\headrulewidth}{0pt}
%\fancyfoot[C]{\fbox{\parbox{250pt}{\vspace{50pt}Page \thepage\ of \pageref{LastPage}}}}
\fancyfoot[C]{\parbox{240pt}{\vspace{50pt}Page \thepage\ of \pageref{LastPage}}}
\pagestyle{fancy}

% positioning adjustments
\setlength{\topmargin}{0.25in}
\setlength{\headheight}{0in}
\setlength{\headsep}{0in}
\setlength{\topskip}{0in}
\setlength{\textwidth}{4.75in}
\setlength{\textheight}{8.5in}
\setlength{\oddsidemargin}{0.25in}
\setlength{\evensidemargin}{0.25in}
\setlength{\footskip}{0.5in}
\setlength{\paperheight}{11in}
\setlength{\paperwidth}{8.5in}

%environments and commands
\newenvironment{list1}{
 \begin{list}{\ding{113}}{%
      \setlength{\itemsep}{0in}
      \setlength{\parsep}{0in} \setlength{\parskip}{0in}
      \setlength{\topsep}{0in} \setlength{\partopsep}{0in} 
      \setlength{\leftmargin}{0.17in}}}{\end{list}}
\newenvironment{list2}{
  \begin{list}{$\bullet$}{%
      \setlength{\itemsep}{0in}
      \setlength{\parsep}{0in} \setlength{\parskip}{0in}
      \setlength{\topsep}{0in} \setlength{\partopsep}{0in} 
      \setlength{\leftmargin}{0.2in}}}{\end{list}}

%begin document
\begin{document}

\name{Timothy J. Crone \vspace*{0.1in}}

\begin{resume}
\section{\sc Contact Information}
\vspace{0.05in}
\begin{tabular}{@{}p{2.5in}p{3in}}
Lamont--Doherty Earth Observatory	& (845) 365-8687 \\ 
Columbia University			& tjc@ldeo.columbia.edu\\         
61 Route 9W				& http://www.fluidcontinuity.org\\
Palisades, NY 10964			& https://github.com/tjcrone\\     
\end{tabular}

\section{\sc Research\\Interests}
I am a marine geophysicist interested in the interplay between large-scale geophysical processes and the microbial biosphere. I am currently studying the nature of fluid flow variability within seafloor hydrothermal systems, with the hope of characterizing the effect of tidal and tectonic forces on hydrothermal convection at mid-ocean ridges. The ultimate goal of this work is to understand how flow variations affect hydrothermal fluxes and subseafloor biological production. I use numerical models and laboratory studies to investigate these processes, and I develop new observational techniques and instrumentation to verify model predictions.

\section{\sc Education}
{\bf University of Washington}, Seattle, Washington, USA\\
\vspace*{-.1in}
\begin{list1}
\item[] Ph.D., Oceanography (Marine Geophysics), June 2007
\begin{list2}
\vspace*{.05in}
\item Dissertation Topic: ``Tidally-Forced Flow Variability Within Mid-Ocean Ridge
Hydrothermal Systems: Models and Measurement Techniques'' 
\item Advisor: William S. D. Wilcock
\end{list2}
\vspace*{.05in}
\item[] M.S., Oceanography (Marine Geophysics), May 2004
\vspace*{.05in}
\item[] B.S., Oceanography (Physical), June 1999
\end{list1}

\section{\sc Honors and\\Awards} 

Lamont--Doherty Earth Observatory Ten-Years of Loyal Service Award (2018) \\
\vspace*{0.5mm}\\
Lamont--Doherty Earth Observatory Postdoctoral Fellowship (2007-2009) \\
\vspace*{0.5mm}\\
Theodore and Marie Sarchin Graduate Fellowship in Oceanography (2006) \\
\vspace*{0.5mm}\\
NASA Space Grant Summer Fellowship (1999) \\
\vspace*{0.5mm}\\
Mary Gates Research Scholarship (1998) \\
\vspace*{0.5mm}\\
Best Undergraduate Presentation at the Puget Sound Research \\ Conference (1998) \\
\vspace*{0.5mm}\\
National Merit Scholarship Commendation (1991) \\

\section{\sc Academic\\Experience}

{\bf Lamont--Doherty Earth Observatory}, Palisades, New York, USA

{\em Lamont Associate Research Professor, Senior Staff} \hfill {\bf January, 2018 -- Present}\\

\vspace{-.3cm}
{\em Lamont Associate Research Professor} \hfill {\bf July, 2014 -- Present}\\

\vspace{-.3cm}
{\em Lamont Assistant Research Professor} \hfill {\bf July, 2010 -- June, 2014}\\

\vspace{-.3cm}
{\em Doherty Assistant Research Scientist} \hfill {\bf January, 2010 -- June, 2010}\\

\vspace{-.3cm}
{\em Postdoctoral Research Fellow} \hfill {\bf September, 2007 -- December, 2009}\\

{\bf University of Washington}, Seattle, Washington, USA

%\vspace{-.2cm}
{\em Research Assistant} \hfill {\bf September, 2000 -- June, 2007}\\
Includes current Ph.D.~research, Ph.D.~and Masters level course work and
research/consulting projects.

{\em Teaching Assistant} \hfill {\bf January, 2001 -- June, 2005}\\
Assisted in teaching several courses covering a broad range of topics. Duties included
grading, providing supplementary lectures, office hours, and web development.
\vspace*{0.05in}
\begin{list2}
\item Envir215 Earth, Air, Water: The Human Context, Spring 2005
\item Ocean410 Marine Geology and Geophysics, Autumn 2004
\item Ocean451 Fluid Dynamics Laboratory, Winter 2002
\item Ocean485 Fluid Dynamics Laboratory, Winter 2001
\end{list2}

\section{\sc Students\\Advised}

Amanda Liu (Real-Time Earth Cloud Computing Intern, 2018)

Gregory Yap (Real-Time Earth Cloud Computing Intern, 2018)

Jean-Arthur Olive (LDEO Postdoctoral Fellow, 2016--2017)

Robert Colgan (LDEO Summer Intern, 2012)

Peter Ferencevych (LDEO Summer Intern, 2012)

Hannah Moore (LDEO Summer Intern, 2009)
\vspace{0.2in}

\section{\sc Committee\\Service}

Member: NSF Ocean Observatories Initiative Facilities Board

Co-Chair: NSF Ocean Observatories Initiative Data Dissemination and Cyber-Infrastructure Committee

Founder: Lamont Real-Time Earth Initiative

Member: Interdisciplinary Earth Data Alliance (IEDA) Advisory Committee

Member: Lamont Strategic Planning Committee

Chair: MG\&G Strategic Planning Committee

Member: LDEO Advisory Committee for Academic Affairs \& Diversity
\vspace{0.2in}

\section{\sc Papers in\\Preparation}

{\bf Crone, T. J.}, A. Marburg, F. Knuth, and D. S. Kelley (In Prep) Using video image analysis and a cabled seafloor camera system to measure bacterial ``floc'' concentrations at Axial Seamount.

{\bf Crone, T. J.}, Kinsey, J., and E. Mittelstaedt (In Prep) Estimating the total heat output of the ASHES hydrothermal field using the autonomous underwater vehicle Sentry.

{\bf Crone, T. J.} and M. Tolstoy (In Prep) A 4-Dimensional Analysis of poroelastically triggered earthquakes at 9$^\circ$50'N East Pacific Rise.

\section{\sc Manuscripts\\in Review}

Olive, J.-A., and {\bf T. J. Crone} (In Review) Smoke without fire: How long can thermal cracking sustain hydrothermal circulation in the absence of magmatic heat? {\em J. Geophys. Res.}.

{\bf Crone, T. J.} and D. R. Bohnenstiehl (In Review) Acoustic evidence of a long-lived gas-driven submarine volcanic eruption in the Bismarck Sea, {\em Geophysical Journal International}.

\section{\sc Refereed\\Publications}

Barreyre, T., J.-A. Olive, {\bf T. J. Crone}, and R. A. Sohn (In Press) Depth-dependent permeability and heat extraction at basalt-hosted hydrothermal systems across mid-ocean ridge spreading rates, {\em Geochem. Geophys. Geosyst.}.

{\bf Crone, T. J.}, M. Tolstoy, J. C. Gibson, and G. Mountain (2017)
Utilizing the \textit{R/V Marcus G. Langseth's} Streamer to Measure the Acoustic Radiation of its Seismic Source in the Shallow Waters of New Jersey's Continental Shelf, \textit{PLoS ONE}, doi:10.1371/journal.pone.0183096.

Tontini, F. C., {\bf T. J. Crone}, C. de Ronde, D. J. Fornari, J. Kinsey, E Mittelstaedt,  and M. Tivey (2016) Crustal magnetization and the subseafloor structure of the ASHES vent field, Axial Seamount, Juan de Fuca Ridge: Implications for the investigation of hydrothermal sites, {\em Geophys. Res. Lett.}, doi:10.1002/2016GL069430.

Mittelstaedt, E., D. J. Fornari, {\bf T. J. Crone}, J. Kinsey, D. Kelley, and M. Elend (2016) Diffuse venting at the ASHES hydrothermal field: Heat flux and tidally modulated flow variability derived from in situ time-series measurements, {\em Geochem. Geophys. Geosyst.}, \linebreak doi:10.1002/2015GC006144.

Abadi, S. H., W. S. D. Wilcock, M. Tolstoy, {\bf T. J. Crone}, and S. M. Carbotte (2015) Sound source localization techniques using a seismic hydrophone streamer and its extension for Baleen whale localization during seismic surveys, {\em J. Acoust. Soc. Am.}, 138, doi: 10.1121/1.4937768.

{\bf Crone, T. J.}, M. Tolstoy, and H. Carton (2014) Estimating shallow water sound power levels and mitigation radii for the {\em R/V Marcus G. Langseth} using an 8 km long MCS streamer, {\em Geochem. Geophys. Geosyst.}, 15, doi:10.1002/2014GC005420.

McNutt, M. K., R. Camilli, {\bf T. J. Crone}, G. D. Guthrie, P. A. Hsieh, T. B. Ryerson, O. Savas, and F. Shaffer (2012) Review of flow rate estimates of the {\em Deepwater Horizon} oil spill, {\em Proceedings of the National Academy of Sciences}, doi:10.1073/pnas.1112139108.

{\bf Crone, T. J.}, M. Tolstoy, and D. F. Stroup (2011) Permeability structure of young ocean crust from poroelastically triggered earthquakes, {\em Geophys. Res. Lett.}, 38, L05305, doi:10.1029/2011GL046820.

Diebold, J., M. Tolstoy, L. Doermann, S. Nooner, S. Webb, and {\bf T. J. Crone} (2010) {\em R/V Marcus G. Langseth} seismic source: Modeling and calibration, {\em Geochem., Geophys., Geosyst.}, 11, Q12012,\\doi:10.1029/2010GC003216.

{\bf Crone, T. J.}, and M. Tolstoy (2010) Magnitude of the 2010 Gulf of Mexico oil leak, {\em Science}, 330, 634, doi:10.1126/science.1195840.

{\bf Crone, T. J.}, W. S. D. Wilcock, and R. E. McDuff (2010) Flow rate perturbations in a black smoker hydrothermal vent in response to a mid-ocean ridge earthquake swarm, {\em Geochem., Geophys., Geosyst.}, 11, Q03012, doi:10.1029/2009GC002926.

Winckler, G., R. Newton, P. Schlosser, and {\bf T. J. Crone} (2010) Mantle helium reveals Southern Ocean hydrothermal venting, {\em Geophys. Res. Lett.}, 37, L05601, doi:10.1029/2009GL042093.

Stroup, D. F., M. Tolstoy, {\bf T. J. Crone}, A. Malinverno, D. R. Bohnenstiehl, and F. Waldhauser (2009) Systematic along-axis tidal triggering of microearthquakes observed at 9$^\circ$50'N East Pacific Rise, {\em Geophys. Res. Lett.}, 36, L18302, doi:10.1029/2009GL039493.

Tolstoy, M., J. Diebold, L. Doermann, S. Nooner, S. C. Webb, D. R. Bohnenstiehl, {\bf T. J. Crone}, and R. C. Holmes (2009) Broadband calibration of the {\em R/V Marcus G. Langseth} four-string seismic sources, {\em Geochem., Geophys., Geosyst.}, 10, Q08011, doi:10.1029/2009GC002415.

{\bf Crone, T. J.}, R. E. McDuff, and W. S. D. Wilcock (2008) Optical plume velocimetry: A new flow measurement technique for use in seafloor hydrothermal systems, {\em Experiments in Fluids}, doi:10.1007/s00348-008-0508-2.

{\bf Crone, T. J.} (2007) Tidally-forced flow variability within mid-ocean ridge hydrothermal systems: Models and measurement techniques, Ph.D. Thesis, School of Oceanography, University of Washington, Seattle, WA, USA.

{\bf Crone, T. J.}, W. S. D. Wilcock, A. H. Barclay, and J. Parsons (2006) The sound generated by mid-ocean ridge black smoker hydrothermal vents, {\em PLoS ONE}, 1(1): e133, doi:10.1371/journal.pone.0000133.

{\bf Crone, T. J.}, and W. S. D. Wilcock (2005) Modeling the effects of tidal loading on mid-ocean ridge hydrothermal systems, {\em Geochem., Geophys., Geosyst.}, 6, Q07001, doi:10.1029/2004GC000905.

\section{\sc Conference\\Publications}

Marburg, A., F. A. Knuth, and {\bf T. J. Crone} (2017) Cloud-accelerated analysis of subsea high-definition camera data, \textit{Oceans 2017 MTS/IEEE}, Anchorage, AK, September 18--21.

Knuth, F., A. Marburg, and {\bf T. J. Crone} (2017) Deriving quantitative metrics from OOI high-definition video data for the purpose of automated QA/QC, \textit{Oceans 2017 MTS/IEEE}, Anchorage, AK, September 18--21.

{\bf Crone, T. J.}, C. C. Ebbesmeyer,
and W. J. Ingraham, Jr. (1998) Dispersion of 1,000 drift cards released over Victoria's sewage outfalls, {\em Puget Sound Research `98 Proceedings}, Washington State Convention and Trade Center, Seattle, Washington, 12--13 March.

Ebbesmeyer, C. C., C. A. Coomes, J. M. Cox, {\bf T. J. Crone}, K. A. Kurrus, E. C. Noah, R. Shuman (1998) Current structure in Elliott Bay, Washington: 1977--1996, {\em Puget Sound Research `98 Proceedings}, Washington State Convention and Trade Center, Seattle, Washington, 12--13 March.

\section{\sc Opinions} I. A. MacDonald, J. Amos, {\bf T. J. Crone}, and S. Wereley (2010) The measure of a disaster, {\em New York Times}, Opinion, 21 May.

\section{\sc Abstracts}

Marburg, A., {\bf T. J. Crone}, F. A. Knuth (2018) Cloud-scaled static scene labeling in subsea high-definition video. Ocean Sciences 2018, Portland, OR, February 11--16.

{\bf Crone, T. J.}, J. C. Kinsey, and E. Mittelstaedt (2017) Estimating the total heat flux from the ASHES hydrothermal vent field using the Sentry autonomous underwater vehicle. Abstract T31C-0639 presented at the 2017 Fall Meeting, AGU, New Orleans, LA, 11--15 December.

Rahman, M., {\bf T. J. Crone}, F. Knuth, C. Garcia, D. C. Soule and R. Fatland (2017) Examining the effect of temperature, pressure, seismicity and diffuse fluid flow on floc events at Axial Seamount. Abstract T33G-04 presented at the 2017 Fall Meeting, AGU, New Orleans, LA, 11--15 December.

Knuth, F., A. Marburg, and {\bf T. J. Crone} (2017) Using image analysis to explore changes in bacterial mat coverage at the base of a hydrothermal vent within the caldera of Axial Seamount. Abstract H31F-1578 presented at the 2017 Fall Meeting, AGU, New Orleans, LA, 11--15 December.

{\bf Crone, T. J.}, F. Knuth, and A. Marburg (2016) Using the OOI Cabled Array HD camera to explore geophysical and oceanographic problems at Axial Seamount. Abstract OS41C-1970 presented at the 2016 Fall Meeting, AGU, San Francisco, CA, 12--16 December.

Wilcock, W. S. D., M. Tolstoy, F. Waldhauser, Y. J. Tan, C. Garcia, A. F. Arnulf and {\bf T. J. Crone} (2016, invited) Earthquake tidal triggering associated with the 2015 eruption of Axial Seamount. Abstract OS44B-03 presented at the 2016 Fall Meeting, AGU, San Francisco, CA, 12--16 December.

Olive, J.-A., {\bf T. J. Crone}, and W. R. Buck (2016) Ultraslow ridge tectonics controlled by deep hydrothermal processes in the absence of a magmatic input. Abstract T32A-04 presented at the 2016 Fall Meeting, AGU, San Francisco, CA, 12--16 December.

{\bf Crone, T. J.},  T. Barreyre, and J.-A. Olive (2015) Depth-dependent permeability of hydrothermal discharge zones: measurements through tidal modulation and implications for mid-ocean ridge heat budgets. Abstract OS43A-2021 presented at the 2015 Fall Meeting, AGU, San Francisco, CA, 14--18 December.

Mittelstaedt, E. L., D. J. Fornari, and {\bf T. J. Crone} (2015) Time Series Measurements of Diffuse Hydrothermal Flow at the ASHES Vent Field Reveal Tidally Modulated Heat and Volume Flux. Abstract OS41B-04 presented at the 2015 Fall Meeting, AGU, San Francisco, CA, 14--18 December.

{\bf Crone, T. J.}, E. Mittelstaedt, and D. Fornari (2014) Using the VentCam and Optical Plume Velocimetry to Measure High-Temperature Hydrothermal Fluid Flow Rates in the ASHES Vent Field on Axial Volcano. Abstract V12B-07 presented at 2014 Fall Meeting, AGU, San Francisco, CA, 15--19 December.

Mittelstaedt, E., {\bf T. J. Crone}, and D. Fornari (2014) Using a New Deep-Sea Camera System to Measure Temporal Variations of the Volume and Heat Flux of Diffuse Hydrothermal Fluids at the ASHES Vent Field. Abstract V12B-08 presented at 2014 Fall Meeting, AGU, San Francisco, CA, 15--19 December.

{\bf Crone, T. J.}, R. A. Sohn, and T. Barreyre (2014) Modeling Seafloor Deformation at the TAG Hydrothermal Field: Feedbacks between Permeability and Poroelastic Fluid Flow? Abstract V21A-4728 presented at 2014 Fall Meeting, AGU, San Francisco, CA, 15--19 December.

Kinsey J., {\bf T. J. Crone}, E. Mittelstaedt, L. Medagoda, D. Fourie, and K. Nakamura (2014) Estimating the Heat and Mass Flux at the ASHES Hydrothermal Vent Field with the Sentry Autonomous Underwater Vehicle. Abstract V21A-4725 presented at 2014 Fall Meeting, AGU, San Francisco, CA, 15--19 December.

Barreyre, T., R. A. Sohn, and {\bf T. J. Crone} (2014) Global synthesis and analysis of deep-sea hydrothermal time-series data: Toward a characterization of the outflow dynamics. Abstract V11E-04 presented at 2014 Fall Meeting, AGU, San Francisco, CA, 15--19 December.

Abadi, S. H., {\bf T. J. Crone}, M. Tolstoy, W. S. D. Wilcock, and S. M. Carbotte (2014) Estimating the range of Baleen whale calls recorded by hydrophone streamers during seismic surveys. Abstract presented at the 167th Meeting of the Acoustical Society of America, Providence, Rhode Island, 5--9 May.

Reitz, M. D., C. P. Stark, C. Hung, B. Smith, E. Grinspun, H. Capart, L. Li, L. Hsu, {\bf T. J. Crone}, and H. I. Ling (2014) Connecting grain-scale physics to macroscopic granular flow behavior using discrete contact-dynamics simulations, centrifuge experiments, and continuum modeling. Abstract presented at the 2014 General Assembly, EGU, Vienna, Austria, 27 April--02 May.

{\bf Crone, T. J.}, M. Tolstoy, and H. D. Carton (2013) Calibration of the {\em Marcus G. Langseth} seismic array in shallow Cascadia waters using the multi-channel streamer. Abstract presented at the 2013 Fall Meeting, AGU, San Francisco, CA, 9--13 December.

Hung, C., H. Capart, {\bf T. J. Crone}, E. Grinspun, L. Hsu, D. Kaufman, L. Li, H. I. Ling, M. D. Reitz, B. Smith, and C. P. Stark (2013) Scaling up debris flow experiments on a centrifuge. Abstract presented at the 2013 Fall Meeting, AGU, San Francisco, CA, 9--13 December.

{\bf Crone, T. J.}, R. E. Colgan, and P. G. Ferencevych (2012) Video image analysis of turbulent buoyant jets using a novel laboratory apparatus. Abstract EP53B-1024 presented at 2012 Fall Meeting, AGU, San Francisco, CA, 3--7 December.

{\bf Crone, T. J.}, R. A. Sohn, and S. C. Webb (2010) Modeling ground surface deformation at the TAG hydrothermal field using feedbacks between permeability and poroelastic flow, {\em Proceedings of the Goldschmidt Conference on Earth, Energy, and the Environment}, Knoxville, TN, 13--18 June 2010.

Ferrini, V., S. Soule, S. M. White, and {\bf T. J. Crone} (2009) Seafloor change and lava emplacement processes associated with the 2005-2006 East Pacific Rise eruptions, {\em Eos Trans. AGU}, 90(52), Fall Meet. Suppl., Abstract V51D-1726.

{\bf Crone, T. J.}, M. Tolstoy, and D. F. Stroup Sumy (2009) Using two-dimensional models of poroelastic fluid flow to constrain the permeability structure of young oceanic crust, {\em Proceedings of the Ridge 2000 Integration and Synthesis Workshop}, St. Louis, MO, 1--3 October 2009.

Stroup, D. F., M. Tolstoy, {\bf T. J. Crone}, A. Malinverno, D. R.  Bohnenstiehl, and F. Waldhauser (2009) Systematic variations in along-axis tidal triggering of microearthquakes observed at 9$^\circ$50'N East Pacific Rise, {\em Proceedings of the Ridge 2000 Integration and Synthesis Workshop}, St. Louis, MO, 1--3 October 2009.

{\bf Crone, T. J.}, (2009) Using two-dimensional models of poroelastic fluid flow to constrain permeability in young oceanic crust, {\em Proceedings of the Marine Geoscience Leadership Symposium}, Washington DC, 23--27 March 2009.

{\bf Crone, T. J.}, M. Tolstoy, and D. F. Stroup (2008) Two-dimensional models of poroelastically-controlled earthquake triggering at the East Pacific Rise, {\em Eos Trans. AGU}, 89(53), Fall Meet. Suppl., Abstract B23F-06.

Tolstoy, M., {\bf T. J. Crone}, F. Waldhauser, D. R. Bohnenstiehl, D. J.  Fornari, and K. Von Damm (2008) Seismic activity associated with temperature perturbations at Bio 9 hydrothermal vent on the East Pacific Rise at 9$^\circ$50'N, {\em Eos Trans. AGU}, 89(53), Fall Meet. Suppl., Abstract B21A-0326.

Stroup, D. F., M. Tolstoy, {\bf T. J. Crone}, A. Malinverno, D. R.  Bohnenstiehl, and F.  Waldhauser (2008) Tidal triggering of microearthquakes constrains permeability at 9$^\circ$50'N East Pacific Rise, {\em Eos Trans.  AGU}, 89(53), Fall Meet. Suppl., Abstract B21A-0324.

Diebold, J., M. Tolstoy, S. Webb, L. Doermann, D. Bohnenstiehl, S. Nooner, {\bf T. J. Crone}, and R. C. Holmes (2008) Calibration of {\em R/V Marcus G.  Langseth} seismic sources, {\em Eos Trans. AGU}, 89(53), Fall Meet. Suppl., Abstract IN13A-1074.

{\bf Crone, T. J.} (2008) Integrating numerical models of poroelastic convection with time-series data at the EPR, {\em Proceedings of the Ridge 2000 East Pacific Rise Integration and Synthesis Workshop}, Cotuit, MA, 26--28 September 2008.

{\bf Crone, T. J.} (2008) Integrating numerical models of poroelastic convection with time-series data at the Endeavour, {\em Proceedings of the Ridge 2000 Endeavour Integration and Synthesis Workshop}, Seattle, WA, 18--19 September 2008.

{\bf Crone, T. J.}, and M. Tolstoy (2008) Using applied fluorescent tracers to study mid-ocean ridge hydrothermal systems, {\em Proceedings of Mantle to Microbe: Integrated Studies at Oceanic Spreading Centers}, Portland, OR, 24--26 March 2008.

Stroup, D. F., M. Tolstoy, {\bf T. J. Crone}, A. Malinverno, D. R.  Bohnenstiehl, and F.  Waldhauser (2008) The relationship between poroelastic effects and tidal triggering of microearthquakes at 9$^\circ$50'N East Pacific Rise, {\em Proceedings of Mantle to Microbe: Integrated Studies at Oceanic Spreading Centers}, Portland, OR, 24--26 March 2008.

{\bf Crone, T. J.}, W. S. D. Wilcock, and R. E. McDuff (2006) Measuring black smoker fluid flow rates using image correlation velocimetry, {\em Eos Trans.  AGU}, 87(52), Fall Meet.  Suppl., Abstract OS31C-1656.

{\bf Crone, T. J.}, W. S. D. Wilcock, and R. E. McDuff (2006) Hydrothermal flow variability: Two new potential measurement techniques, {\em Proceedings of the Ridge Theoretical Institute}, Mammoth Lakes, CA, 25--30 June 2006. 

{\bf Crone, T. J.}, and W. S. D. Wilcock (2005) The acoustic signature of high-temperature deep-sea hydrothermal vents, {\em Eos Trans. AGU}, 85(52), Fall Meet. Suppl., Abstract HT31A-0486.

Wilcock, W. S. D., {\bf T. J. Crone}, and R. E. McDuff (2003) Tidal variations in fluid discharge velocities at mid-ocean ridge hydrothermal systems: A critical measurement, {\em Geophysical Research Abstracts}, Vol. 5, Proceedings of the EGS-AGU-EUG Joint Assembly, Nice, France, 6--11 April 2003.

{\bf Crone, T. J.}, and W. S. D. Wilcock (2002) Modeling the effects of tidal loading on hydrothermal discharge at mid-ocean ridges, {\em Eos Trans. AGU}, 83(47), Fall Meet.  Suppl., Abstract H21B-0808.

{\bf Crone, T. J.}, and W. S. D. Wilcock (2002) Modeling the effects of tidal loading on hydrothermal discharge at mid-ocean ridges, {\em Proceedings of the InterRidge Theoretical Institute}, Pavia, Italy, 9--13 September 2002.

Cherkaoui, A. S., {\bf T. J. Crone}, and W. S. D. Wilcock (1998) Spatial and temporal characteristics of high Rayleigh number thermal convection in an open-top Hele--Shaw cell, {\em Eos Trans. AGU}, 79(45), Fall Meet. Suppl., Abstract T32A-02.

%\section{\sc Pending\\Proposals}

\section{\sc Funded\\Proposals}

Abernathey, R. P., T. J. Crone, and C. Zappa (2017) Real-Time Earth Cloud Computing Internship Program, {\bf \$233,600}.

Crone, T. J. (2017) Collaborative Research: Cloud-Capable Tools for MG\&G-Related Image Analysis of OOI HD Camera Video, NSF OCE-1801880, {\bf \$19,514}.

Crone, T. J. (2017) Real-Time Power: A Thermoelectric System for Powering Seafloor Instrumentation, NSF OCE-1820547, {\bf \$149,611}.

Abernathey, R. P., T. J. Crone, and C. Zappa (2017) AI for Earth: Real-Time Earth Proposal for Microsoft Azure Cloud Computing Credits, {\bf \$10,000}.

Huber, J., T. J. Crone, and D. Kelley (2016) Drilling into Young Oceanic Crust for Subseafloor Observations at Axial Seamount, IODP USSSP, {\bf \$35,660}.

Crone, T. J. (2016) Collaborative Research: Cloud-Capable Tools for MG\&G-Related Image Analysis of OOI HD Camera Video, NSF OCE-1700923, {\bf\$97,822}

Crone, T. J. (2015) Supplement to: Collaborative Research: Developing New Instrumentation to Accurately Measure Heat and Mass Flux of Hydrothermal Fluids, OCE-1503674, {\bf\$20,652}.

Crone, T. J. (2014) Collaborative Research: RAPID: Testing High Temperature Subseafloor Tracers and Optical Communication Networks at Axial Seamount Using Available DSV Alvin Bottom Time, NSF OCE-1445723, {\bf\$18,894}.

Stark, C. P., T. J. Crone, E. Grinspun, H. I. Ling, M. D. Reitz (2013) Hazards SEES Type 1: Predicting landslide runout and granular flow hazard: Enhanced-g centrifuge experiments, contact dynamics model development, and theoretical study, NSF EAR-1331499, {\bf\$299,748}.

Crone, T. J. (2013) EAGER: Collaborative Research: Using available Sentry AUV aboard R/V Atlantis to measure hydrothermal heat flux at Axial and Main Endeavour Fields, NSF OCE-1338236, {\bf\$17,081}.

Crone, T. J. (2012) Using the R/V Langseth streamer to verify a transmission loss model's predicted sound level as a function of distance from the vessel, {\bf\$34,837}. 

Crone, T. J. (2011) Collaborative Research: Developing new instrumentation to accurately measure heat and mass flux of hydrothermal fluids, NSF OCE-1131455, {\bf\$373,982}.

Crone, T. J. (2011) RAPID: Deploying VentCam as a data collection base station for the Axial Seamount Sensorbot Array, NSF OCE-1147104, {\bf\$36,724}.

Crone, T. J. (2010) SGER: A seafloor camera system for flow rate measurements in black smoker vents, NSF OCE-1036284, {\bf\$44,985}.

Crone, T. J. (2009) Using numerical models of poroelastic fluid flow to constrain the permeability structure of young oceanic crust, NSF OCE-0928181, {\bf\$152,909}.

Crone, T. J. (2009) A seafloor camera system for flow rate measurements in black smoker vents, NSF OCE-0917955, {\bf\$97,386}.

Crone, T. J. (2008) A high-speed self-contained video camera system for optical plume velocimetry, Paros-PGI Observatory Technical and Innovation Center (OTIC), {\bf\$26,160}.

McDuff, R. E. (2006) Measuring black smoker flow rates using image correlation velocimetry, NSF OCE-0623285, {\bf\$33,895}.

Crone, T. J., and J. D. Parsons (2004) Listening to hydrothermal vents: using measurements of turbulent acoustic energy to estimate black smoker fluid velocities, University of Washington Royalty Research Fund, {\bf\$28,600}.

Crone, T. J. (1998) A drift card study in Victoria Bight, The Explorer's Club, {\bf\$3000}.

\section{\sc Funding\\Summary}
{\bf Career Total: \$1,735,060}
\vspace{0.2in}

\section{\sc Selected\\Presentations}

``Real-Time Remote Investigation of a Mid-Ocean Ridge Hydrothermal Vent using a Deep-Sea High-Definition Camera'', National Ocean Exploration Forum, Qualcomm Institute, 21 October 2017

``Measuring and Modeling Fluid Flow within Mid-Ocean Ridge Hydrothermal Systems'', Tenure review presentation, 22 March 2017.

``Real-Time Earth'', Presentation to the Columbia Office of Alumni and Development, 20 September 2016.

``Real-Time Earth: Realizing the Future of Earth Science'', LDEO Open House public lecture, 8 October 2016.

``Real-Time Earth: Strategic Initiative Update'', Presentation to the Lamont Advisory Board, Columbia Club of New York, 14 September 2016.

``Inner Space: Doing Science in the Deep Ocean'', The Explorer's Club, 24 October 2015, {\bf Invited}.

``Mid-Ocean Ridge Hydrothermal Systems: How Do They Work?'', Sparkstravaganza, 24 June 2015.

``The Spaghetti Monster: A very small aperture hydrophone/geophone array to map hydrothermal fluid pathways using beam forming'', OOI NOVAE Workshop, 21 April 2015.

``Modeling at Axial Seamount: Connecting measurements to understanding'', OOI NOVAE Workshop, 20 April 2015, {\bf Invited}.

``I (heart) poroelasticity'', Geodynamics Lecture, Woods Hole Oceanographic Institution, 5 March 2015, {\bf Invited}.

``Using the VentCam and Optical Plume Velocimetry to measure high-temperature hydrothermal flow rates in the ASHES vent field on Axial Volcano'', AGU Fall Meeting, 15 December 2014 {\bf Invited}.

``Real-Time Earth: A New Strategic Initiative'', Presentation to the Lamont Advisory Board, LDEO, 10 December 2014.

``Emerging technologies for measuring fluid flow rates and properties in mid-ocean ridge hydrothermal systems'', Presentation to the Schmidt Ocean Institute, LDEO, 20 September 2013.

``Measuring the Deepwater Horizon Oil Leak Using Optical Plume Velocimetry'', LDEO Summer Intern Presentation, 5 July 2012.

``Measuring the Size of the Deepwater Horizon Oil Leak: Science in the Media Spotlight'', School of International and Public Affairs Lecture, 25 June 2012, {\bf Invited}.

``Way of the Water: BP Oil Spill Panel'', Barnard College, 10 April 2012, {\bf Invited}.

``Lessons learned: Estimating the flow from the Deepwater Horizon oil leak using optical plume velocimetry'', Sustainable Development Seminar Series, Earth Institute of Columbia University, 18 October 2011, {\bf Invited}.

``Using the VentCam flow measurement system to measure flow rates in hydrothermal systems at Axial Seamount'', Axial Seamount RSN Science Workshop, 5 October 2011. 

``Measuring the size of the Deepwater Horizon oil leak: Science in the media spotlight'', {\em R/V Thomas G. Thompson}, 27 August, 2011.

``Measuring the Deepwater Horizon oil leak using optical plume velocimetry: Science in the media spotlight'', School of International and Public Affairs Lecture, 27 June 2011, {\bf Invited}.

``Estimating the size of the Deepwater Horizon oil leak using optical plume velocimetry'', Rhodes College, 23 February 2011, {\bf Invited}.

``Estimating the size of the Deepwater Horizon oil leak using optical plume velocimetry'', LDEO Colloquium, 21 January 2011 {\bf Invited}.

``Estimating the size of the Deepwater Horizon oil leak using optical plume velocimetry'', AGU Fall Meeting, 14 December 2010.

``Measuring the size of the Deepwater Horizon oil leak: Science in the media spotlight'', Old Dominion University, 11 November 2010, {\bf Invited}.

``Measuring the size of the Deepwater Horizon oil leak: Science in the media spotlight'', Noon Balloon Lecture, Columbia University, 5 October 2010, {\bf Invited}.

``Estimating the magnitude of the spill'', LDEO Director's Circle Lecture, 25 September 2010, {\bf Invited}.

``The 2010 Deepwater Horizon oil release: An oceanographic perspective'', Ocean 101 Lecture, Columbia University, 14 September 2010, {\bf Invited}.

``Measuring and modeling fluid flow in hydrothermal vents and other such deep-water places'', {\em R/V Thomas G. Thompson}, 19 August, 2010.

``Poroelastically triggered earthquakes at the East Pacific Rise: A new view of permeability in young oceanic crust'', Boise State University, 26 April 2010, {\bf Invited}.

``The sounds and sights of hydrothermal vents: Measuring flow using passive acoustics and optical image analysis'', {\rm R/V Atlantis}, 31 December 2009.

``Using two-dimensional models of poroelastic fluid flow to constrain the permeability structure of young oceanic crust'', Ridge 2000 Integration and Synthesis Workshop, 2 October 2009, {\bf Invited}.

``Modeling and measuring fluid flow in mid-ocean ridge hydrothermal systems'', OOI Regional Scale Nodes Office, 11 August 2009, {\bf Invited}.

``Report on the Marine Geoscience Leadership Symposium'', Lamont Leadership Forum, 4 June 2009, {\bf Invited}.

``Using two-dimensional models of poroelastic fluid flow to constrain permeability in young oceanic crust'', Marine Geoscience Leadership Symposium, 23 March 2009.

``Two-dimensional models of poroelastically-controlled earthquake triggering at the East Pacific Rise'', AGU Fall Meeting, 16 December 2008.

``Ocean circulation within Earth's crust'', Earth Institute Postdoc Orientation, 12 Sept 2008, {\bf Invited}.

``Poroelasticity and its implications for subseafloor processes'', WHOI G\&G Seminar, 17 June 2008, {\bf Invited}.

``Subseafloor fluid flow: Making measurements of a keystone process'', WHOI Guest Seminar, 16 June 2008, {\bf Invited}.

``Optical flow: Using image analysis to measure hydrothermal fluid fluxes'', LDEO Series on Image Analysis, 5 June 2008, {\bf Invited}.

``Hydrothermal processes at mid-ocean ridges'', LDEO Summer Intern Orientation, 4 June 2008, {\bf Invited}.

``I (heart) poroelasticity - Part 2: Models , measurements , and implications of poroelastic processes in mid-ocean ridge hydrothermal systems'', LDEO MG\&G Seminar, 14 May 2008.

``I (heart) poroelasticity - Part 1: Theory and applications '', LDEO Geodynamics Seminar, 2 May 2008.

``Models and measurements of tidally-forced flow variability within mid-ocean ridge hydrothermal systems'', University of Washington Applied Physics Lab, 15 November 2007, {\bf Invited}.

\section{\sc Synergistic\\Activities}
Member: American Geophysical Union

Lead Organizer: LDEO Joint MG\&G--SG\&T Seminar Series (2008-2009)

Educational Outreach:
\begin{list2}
\item LDEO High School Intern Program Mentor
\item LDEO Summer Intern Program Mentor
\item Ocean Inquiry Project Instructor (www.oceaninquiry.org)
\item School of Oceanography Outreach Volunteer
\item Developed several hands-on fluid dynamics experiments for the School of Oceanography's Open House 
\end{list2}
Selected Conferences and Workshops: 
\begin{list2}

\item OOI NOVAE Workshop, Seattle, WA, 21 April 2015.
\item Marcus Langseth Science Oversight Committee (MLSOC) Annual Meeting, San Francisco, CA, 14 December 2014.
\item UNOLS Deep Submergence Science Committee (DeSSC) Annual Community Meeting, San Francisco, CA, 14 December 2014.
\item UNOLS Deep Submergence Science Committee (DeSSC) New User Program (Session Leader), San Francisco, CA, 13 December 2014.
\item Earthcube End User Domain Workshop for Deep Sea Processes and Dynamics, Narragansett, RI, 2013.
\item Science Planning Meeting for OOI Regional Scale Node at Axial Seamount, Seattle, WA, 2011.
\item Limits to Life C-DEBI Workshop, Redondo Beach, CA, 2011.
\item Deepwater Horizon Oil Spill Scientific Symposium, Baton Rouge, LA, 2010.
\item Ridge 2000 Community Meeting, Portland, OR, 2010.
\item Marine Geoscience Leadership Symposium, Washington DC, 2009.
\item Ridge 2000 East Pacific Rise Integration and Synthesis Workshop, Cotuit, MA, 2008.
\item Ridge 2000 Endeavour Integration and Synthesis Workshop, Seattle, WA, 2008.
\item Ridge 2000 Mantle to Microbe: Integrated Studies at Oceanic Spreading Centers, Portland, OR, 2008.
\item Ridge Theoretical Institute, Mammoth Lakes, California, 2006.
\item Ridge 2000 Progress and Planning Meeting, Vancouver, BC, Canada, 2005.
\item Cyprus Field School and Conference, Cyprus, 2005.
\item InterRidge Theoretical Institute, Pavia, Italy, 2002. 
\end{list2}

\section{\sc Professional\\Experience}

{\bf Timothy J. Crone Consulting, LLC.}, New York, NY\\
\textit{Principal} \hfill {\bf April 2011 -- Present}\\
Scientific consulting services, video image analysis, hydroacoustic monitoring,
data analysis, visualization, reporting, expert witness testimony.

{\bf Global Remote Sensing, LLC.}, Seattle, Washington\\
{\em Scientific Consultant} \hfill {\bf March 2006 -- July 2006}\\
Geoacoustics application development, experiment design and assessment, acoustic data
analysis and visualization.

{\bf The Glosten Associates, Inc.}, Seattle, Washington\\
{\em Scientific Consultant} \hfill {\bf October 2004 -- December 2004}\\
Dye study design, data analysis and visualization.

{\bf Evans-Hamilton, Inc.}, Seattle, Washington\\
{\em Oceanographer II} \hfill {\bf May 1996 -- May 2000}\\
Oceanographic consulting on a wide range of physical oceanographic projects for government
and industry, including environmental remediation, outfall planning, sediment
transport monitoring, and current measurements and modeling for exploration/production
operations.

\section{\sc Seagoing\\Experience}

Jul 2014: {\bf\em R/V Atlantis}
\begin{list2}
\item Chief Scientist
\item Axial Seamount, field testing VentCam and DEMS camera using {\bf\em DSV Alvin}
\end{list2}

\vspace*{-.1in}
Aug 2011: {\bf\em R/V Thomas G. Thompson}
\begin{list2}
\item Axial Seamount, deploying VentCam as a data collection base station for the Sensorbot array using {\bf\em ROV ROPOS}
\end{list2}

\vspace*{-.1in}
Aug 2010: {\bf\em R/V Thomas G. Thompson}
\begin{list2}
\item Axial Seamount, field testing of VentCam flow meter using {\bf\em ROV Jason}
\end{list2}

\vspace*{-.1in}
Dec 2009: {\bf\em R/V Atlantis}
\begin{list2}
\item East Pacific Rise, field testing of VentCam flow meter, participant on {\bf\em DSV Alvin} dives 4577 and 4584
\end{list2}

\vspace*{-.1in}
Jan 2008: {\bf\em R/V Langseth}
\begin{list2}
\item Gulf of Mexico, air-gun calibrations
\end{list2}

\vspace*{-.1in}
Aug/Sep 2005: {\bf\em R/V Atlantis}
\begin{list2}
\item Endeavour Segment, deployment of vent hydrophone, participant on {\bf\em DSV Alvin} dive 4137
\end{list2}

\vspace*{-.1in}
Sep 2004: {\bf\em R/V Thomas G. Thompson}
\begin{list2}
\item Endeavour Segment, deployment of hydrophone using {\bf\em ROV ROPOS}
\end{list2}

\vspace*{-.1in}
Aug 2004: {\bf\em R/V Point Lobos}
\begin{list2}
\item Monterey Canyon, two cruises, deployment/recovery of SlowFlow meter using {\bf\em ROV Ventana}
\end{list2}

\vspace*{-.1in}
May 2004: {\bf\em R/V Atlantis}
\begin{list2}
\item Northeast Pacific, acoustic telemetry mooring deployment
\end{list2}

\vspace*{-.1in}
Sep 2000: {\bf\em R/V Atlantis}
\begin{list2}
\item Endeavour Segment, participant on {\bf\em DSV Alvin} dive 3616
\end{list2}

\vspace*{-.1in}
Jul 2000: {\bf\em R/V Calabar Carrier}
\begin{list2}
\item Bight of Benin, Nigeria, current meter recovery/redeployment
\end{list2}

\vspace*{-.1in}
Mar 2000: {\bf\em R/V Calabar Carrier}
\begin{list2}
\item Bight of Benin, Nigeria, current meter deployment
\end{list2}

\vspace*{-.1in}
1998--2000: {\bf\em R/V Reflux}
\begin{list2}
\item Puget Sound, numerous cruises in Budd Inlet, Elliot Bay, Hylebos Waterway, and
Willipa Bay
\end{list2}

\vspace*{-.1in}
Apr 1999: {\bf\em R/V Thomas G. Thompson}
\begin{list2}
\item Puget Sound, CTD tow-yos
\end{list2}

\end{resume}
\end{document}
